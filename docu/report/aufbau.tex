\chapter{Passiv Radar Setup}\label{sct:setup}
Das Passivradar setzt sich aus Hardwarekomponenten wie Antennen und Empfängern sowie der Softwarekomponenten, die nachher die Empfangen-Signale auswertet und eine Vorhersage über die Lokalisation des Objekts und mit welcher Geschwindigkeit es sich bewegt. Dieses Kapitel beschäftigt sich mit den Komponenten, die in diesem Projekt eingesetzt wurden. Dies beinhaltet auch unsere Algorithmen für die Synchron des Referenzsignals mit dem reflektierten Signal sowie weitere Algorithmen zur Bestimmung der Position und Geschwindigkeit.
\section{Hardware}
Die Hardware eines Passivradars setzt sich zusammen aus Empfängern und Antennen. Verwendet werden in diesem Projekt zwei Pluto SDRs sowie zwei Yagi Antennen für das Empfangen des Referenzsignals sowie des reflektierten Signals. Außerdem noch zur Hardware gehört eine Externe Clock zur Synchronisation der beiden hier wird einen OCXO Oscillator verwendet.
\subsection{ADALM-Pluto SDR}\label{sct:sdr}
Bei dem hier verwendeten SDR handelt es sich um ein ADALM
PLUTO SDR. Der Hauptgrund, warum sich in diesem Versuchsaufbau für dieses Gerät entschieden wurde ist die Bandbreite dieses Gerätes die bei bis zu 20 MHz liegt. Die Bandbreite des Signals was hier für Passiv Radar verwendet wird beträgt  5 MHz was einige SDR nicht aufbringen können.  Aufs weitere besitzt das SDR eine Frequenz Abdeckung von 325 KHz bis zu 3.8 MHz. Die weiteren Daten können in der Tabelle~\ref{table:sdr} abgelesen werden. 

\begin{table}
    \centering
        \begin{tabular}[h]{rl}
            Empfänger & 325 KHz - 3.8MHz Frequenz Abdeckung, 200kHz - 20 MHz Bandbreite\\
            Sender  & 325 KHz - 3.8MHz Frequenz Abdeckung, 200kHz - 20 MHz Bandbreite\\
            Anschlüsse & USB 2.0 OTG, USB power adapter\\
            Kompatibilität & MATLAB, Simulink, GNU Radio, Python, und weitere\\
        \end{tabular}
    \caption{Daten des ADALM PlutoSDR}\label{table:sdr}
\end{table}

\subsubsection{Synchronität}\label{sct:Oscillator}
Zur Aufnahme des Referenzsignals als auch des reflektierten Signals werden jeweils ein Pluto-SDR benötigt, diese zwei müssen nun synchron betrieben werden, um das Referenzsignal zu dem reflektierten Signal zu ordnen zu können. Dies wird erreicht durch Daisy Chaining  der beiden Uhren der SDRs und einem OCXO Oscillator als externen Uhr. Die Abbildungen ~\ref{fig:Pluto} zeigen dann den fertigen Aufbau der Beiden SDRs und im Schaltplan in Abbildung ~\ref{fig:Clock} ist zur erkennen wie die Uhr der jeweiligen SDRs aufgebaut ist.

\begin{figure}
    \centering
    \includegraphics[width=\textwidth]{images/Schaltplan_Clock.png}
    \caption{Schaltplan der Clock des Pluto SDRs} \label{fig:Clock}
\end{figure}

\begin{figure}
    \centering
    \begin{subfigure}[Pluto mit Clock]{0.3\textwidth}
        \includegraphics[width=\textwidth]{images/Pluto_1.jpeg}
        \caption{PlutoSDR, OCXO Oscillator}
    \end{subfigure}
    \begin{subfigure}[Daisy Chaining]{0.3\textwidth}
        \includegraphics[width=\textwidth]{images/Pluto_2.jpeg}
        \caption{Daisy Chaining}
    \end{subfigure}
    \begin{subfigure}[Gesamter Aufbau]{0.3\textwidth}
        \includegraphics[width=\textwidth]{images/Pluto_4.jpeg}
        \caption{Der gesamte Aufbau}
    \end{subfigure}
    \caption{Aufbau der Hardware mit den beiden Plutos und der Clock} \label{fig:Pluto}
\end{figure}

\subsection{Antenne}
In diesem Aufbau werden zwei Antennen die für DVB-T gedacht sind verwendet. Die Antenne ist ein Yagi Antenne mit 43 Elementen wie man im Abbildung ~\ref{fig:antenne} sieht die im Frequenzbereich von 470 bis 862 MHZ arbeitet, was für unseren Anwendungsfall sehr gut geeignet ist. Die weiteren Daten zur Antenne stehen in der Tabelle ~\ref{table:antenne}.

\begin{table}
    \centering
        \begin{tabular}[h]{rl}
            Antenne         & SKT SL43-01 UHF 43            \\
            Antennengewinn  & 11..13 dB                     \\
            Frequenzbereich & 470-862 MHz                   \\
            Halbwertsbreite & horiz. 30...40°/ver. 35...50° \\
        \end{tabular}
    \caption{Daten der SKT SL43-01 UHF 43 Antenne}\label{table:antenne}
\end{table}

\begin{figure}
    \centering
    \includegraphics[width=\textwidth]{images/antenne.png}
    \caption{SKT SL43-01 UHF 43 Antenne}\label{fig:antenne}
\end{figure}
\section{Signal}
Das hier verwendete Signal beruht auf der LTE Technik. Long Term Evolution oder kurz LTE und war vor allem dadurch erfolgreich, dass eine neue Übertragungstechnik verwendet wurde, mit der das Multipath-Fading-Effekt umgangen wurde. Der Multipath-Fading-Effekt tritt auf, wenn die Länge eines Übertragungsschrittes verkleinert wird, was bei einer Verzögerung des Signals dazu führen kann, dass sich die einzelnen Übertragungsschritte überlappen. Die Technik, die dem gegen wirkt, hat die Bezeichnung Orthogonal Frequenz Division Multiplexing (OFDM) Technik. Hierbei wird ein schneller Datenstrom in viele kleine langsame unterteilt, da so kein Multipath-Fading-Effekt auftritt, nach dem Übertragen wird Datenstrom wieder zusammengeführt. OFDM wird in einem der unter Kapitel noch genauer erläutert werden. 
Die Bandbreite von LTE ist recht variable wählbar und liegt zwischen 1,25 MHz und 20 MHz, in der bei diesem Projekt auf LTE verwendeten Übertragungstechnik liegt, die Bandbreite bei 5 MHz. Die Wahl der Bandbreite hängt hier sehr davon ab, was das Ziel für die Applikation ist, theoretisch können aber bei einer Bandbreite von 20MHz und sehr guten Übertragungsbedingungen Datenraten von über 100Mbit/s erreicht werden. Weitere nennenswerte Änderungen waren die Einführung von Multiple Input Multiple Output (MIMO) Übertragungen sowie der Fokus auf das paketvermittelnde Internet-Protokoll (IP).~\cite[S.205f]{Sauter2018}
\subsection{Aufbau von LTE}
Ein LTE-Signal besteht aus 504 unterschiedlichen Zellidentitäten auf physikalischer Ebene, diese sind in 168 unterschiedliche physikalischer Gruppen unterteilt. Eine Zellidentität setzt sich zusammen aus Identifikationsnummer der Gruppe sowie der Identifikationsnummer der jeweiligen physikalischen Ebene in der physikalischen Gruppe zusammen. Somit berechnet sich die Zellidentität wie folgt: $$N_{ID}^{cell}=3N_{ID}^{(1)}+N_{ID}^{(2)}$$ Hierbei gilt, dass jede Gruppe eine Nummer $N_{ID}^{(1)}$ zur Identifikation im Bereich von 0 bis 167 besitzt. Außerdem liegt die Identifikationsnummer $N_{ID}^{(2)}$ der physikalischen Ebene in der physikalischen Gruppe im Bereich von 0 bis 2.
Zur Synchronisation des Signals benutzt LTE sowohl ein Primary Synchronisation Signal (PSS) als auch ein Secondary Synchronisation Signal (SSS).~\cite[S.~180]{etsi2021136}

\subsubsection{Framestrukturtyp}
Der Framestrukturtyp 1, der für Vollduplex und Halbdublex FDD gilt, hat eine Länge von 10 ms und besteht aus 10 Subframes, die jeweils eine Länge von $1ms$ haben und von 0 bis 9 durchnummeriert sind. Bei Subframes, die die Frequenz $\Delta f=2.5kHz$, $\Delta f=7.5kHz$ oder $\Delta f=15kHz$ besitzen wird ein Subframe in zwei Slots unterteilt, wo die länge jedes Slots $0.5ms$ beträgt. Bei einer Subframefrequenz von $\Delta f=1.25kHz$ besteht der Subframe aus keinen Slots und seine Länge beträgt $1ms$. Wenn die Übertragungsfrequenz bei $\Delta f=0.37kHz$ liegt, besitzt ein Slot eine Länge von $3ms$. Eine Periode ist $40ms$ lang und besteht dem nach aus 12 Slots, die von 0 bis 13 durchnummeriert sind. Die übertragung beginnt somit dann bei jedem vierten Frame. Außerdem stehen für FDD 10 Subframes, 20 Slots oder bis 60 Subslots zur Verfügung.

\subsubsection{Primary Synchronisation Signal (PSS)}
 PSS wird in jeden Frame zweimal übertragen, und zwar im ersten und 10 Slot. Innerhalb jedes Slots wird das PSS im letzten OFDM Symbol übertragen. Was UE damit erreicht mit dem PSS erreicht ist eine Subframe Synchronisation, eine Slot Synchronisation sowie eine Symbol Synchronisation. Außerdem kann die Mitte des jeweiligen Kanals bestimmt werden in der Frequenz Domain. Ebenfalls wird die passenden PCI aus den drei verschiedenen PCI erkannt.
 Die Sequenz des PSS wird generiert durch die Zadoff-Chu erzeugt. Zadoff-Chu: $$ 	\operatorname{d_u}(n)=\begin{cases} e^{-j\frac{\pi un(n+1)}{63}}, & n=0,1,...,30 \\ e^{-j\frac{\pi u(n+1)(n+2)}{63}}, & n=31,32,...,61 \end{cases}$$~\cite[S.~181]{etsi2021136}. Bei einer MBMS-Zelle, die die Rahmenstruktur 1 verwendet, wird das Synchronsignal im Slot 0 in jedem vierten Subframe übertragen.
\subsubsection{Secondary Synchronisation Signal (SSS)}
Die für den sekundären Synchronisationssignal verwendete Sequenzen d(0),…,d(61) von zwei verknüpften Binärsequenzen der Länge 31. Die verknüpften Sequenzen werden mit einer Verschlüsselungssequenz aus dem ersten Synchronisationssignal verwurschtelt. Die Zuordnung der Sequenz zu den Resource-Elementen hängt von der Framestruktur ab. Bei einem Subframe des Framestrukturtyp 1 wird für das sekundäre Synchronisationssignal derselbe Antennenanschluss wie für das erstes verwendet. ~\cite[S.~183]{etsi2021136} 

\subsection{OFDM}
Wie schon thematisiert wird bei LTE die OFDM Technik verwendet, in diesem Unterkapitell wird nun auf die Technik genauer eingegangen. Nochmal zu Wiederholung durch das Orthogonales Frequenzmultiplexverfahren OFDM Technik ist es möglich mehr Daten gleichzeitig zu übertragen dafür wird das Signal aufgeteilt und auf unterschiedlichen Frequenzbändern(engl.: \textit{subcarriers}) übertragen.

Die Signalübertragung eines OFDM besteht aus drei Phasen: Preamble, Header und den Daten. Durch die Preamble wird Zeit Synchronisation, die Offset sowie die Kanalschätzung durchgeführt. Das Preamble besteht aus einer langen und kurzen Sequenz, die kurze ist zur Bestimmung der Zeit die lange ist für die restlichen Bestimmungen. Die kurze Sequenz besteht aus 10 periodischen Segmenten wo jedes die gleiche Abtastung von 16 Abtastungen besitzt. Die lange Sequenz besteht dagegen aus zwei OFDM-Symbolen. Hier folgen auf Schutzintervalle zwei FTT Intervalle.

Der Header eines OFDM besteht aus 5 Feldern der Datenrate (4 Bits), einem Reservierten Bit (1 Bit), der Länge (12 Bits), einem Parität Bit (1 Bit) und dem Schwanz (6 Bits). Beim Datenrate Feld kann hier zwischen 16 Kombinationen gewählt werden. Das Längenfeld gibt Information über die tatsächliche Länge in Bytes der Daten.

Das Datenformat besteht aus vier Felder Service, Nachricht, dem Schwanz und Pad. Die Aufgabe des Padfeldes ist es, die Länge der Daten so anzupassen, dass die Datenlänge ist ein ganzzahliges Vielfaches von N dbps. Die gesamten Daten Länge ergibt sich ausfolgender Gleichung: $$N_{data}=N_{dbps}(Ceiling(\frac{16+8N_{Nachricht}}{N_{dbps}}))$$
wobei $N_{Nachricht}$ die Länge der Nachricht entspricht, die auch im Header übergeben wird. 16 entspricht der Anzahl an Service Bytes und 6 entspricht der Anzahl an Schwanz Bits. Die Anzahl an Pad hängt von der Daten Länge ab und berechnet sich wie folgt: $$N_{pad}=N_{data}-(16-8N_{Nachricht}+6)$$ ~\cite[S.49ff]{Liu2019}

\section{Software}
\subsection{SDR-angel}
Die Aufnahmen in diesem Projekt wurden mit der Open-Source-Software SDR-Angel gemacht. Mit der auch beide SDRs parrallel aufgezeichnen werden können.
\subsection{Signalverarbeitung}
\subsubsection{Ambiguity Funktion}\label{sct:ambiguity_function}
\subsubsection{Clean Algorithmus}
