\chapter{Durchführung}
In diesem Kapitel wird auf die Durchführung der Messungen mit dem Passivradar eingegangen. Dabei wird erläutert werden, wo sich die Quelle des verwendeten Signals befindet und wo sich der Messpunkt befindet und nach welchen Kriterien er ausgewählt wurde. Außerdem wird der gesamte fertige Messaufbau dargestellt.
\section{Punkt des Signals}
Als Beleuchter dient der Stuttgarter Fernsehturm. Dieser liegt im Süden von Stuttgart und ist in Abbildung~\ref{fig:Fernsehturm} zu sehen. Das 5G Broadcast-Signal wird vom Stuttgarter Fernsehturm in Richtung der A8 ausgesendet. Der Broadcast ist Teil eines Pilotprojektes, gefördert durch einen Zusammenschluss aus Südwestrundfunk (SWR), Kathrein Broadcast GmbH,  DFMG  Deutsche Funkturm GmbH, Porsche AG, Rohde \& Schwarz GmbH \& Co. KG, TU Braunschweig und Telekom Deutschland GmbH. Ziel des Projektes ist es, die Nutzung von 5G Broadcast für Anwendungen in Fahrzeugen zu erproben. Dafür werden vier Services übertragen: Zwei Fernsehprogramme, die ARD/SWR Mediathek und ein Reiseführer~\cite{5GMAG2020}.
\begin{figure}
    \centering
    \includegraphics[width=\textwidth]{images/Fernsehturm.jpg}
    \quelle{\cite{TURM2021}}
    \caption{Stuttgarter Fernsehturm}\label{fig:Fernsehturm}
\end{figure}

\section{Messpunkt}
Der Ort der Feldmessungen ist so gewählt, dass sowohl der Stuttgarter Flughafen, als auch der Stuttgarter Fernsehturm für das Radar sichtbar sind. Da der Aufbau nicht im zusammengebauten Zustand transportfähig ist, muss dieser vor Ort montiert werden. Dies kann wertvolle Zeit kosten, weshalb der Messplatz bewusst in der Nähe des Studentenwohnheims Geschwister-Scholl-Str.\@ gewählt wurde. Unweit des Messplatzes befindet sich der Segelflugplatz Esslingen, jedoch konnten während den Feldmessungen keine Segelflugstarts oder Landungen beobachtet werden. Der Messaufbau wurde daher \SI{300}{\metre} südlich mit Blick auf das Tal verlegt. Die genaue Position ist in Abbildung~\ref{fig:Einflugschneise} und in Abbildung~\ref{fig:Maps} auf Google Maps zu erkennen.

\begin{figure}
    \centering
    \includegraphics[width=\textwidth]{images/Einflugschneise.jpg}
    \caption{Einflugschneise und Stuttgart Fernsehturm}\label{fig:Einflugschneise}
\end{figure}

\begin{figure}
    \centering
    \includegraphics[width=\textwidth]{images/Maps_Messpunkt.png}
    \caption{Possition des Messpunkt}\label{fig:Maps}
\end{figure}

\section{Messaufbau}
Der Messaufbau besteht aus den Komponenten die im Kapitel~\ref{sct:setup} auf S.\pageref{sct:setup} erläutert wurden. Der fertige Aufbau ist in Abbildung~\ref{fig:Messaufbau} zu sehen. Hier werden die beiden SDRs~\ref{sct:sdr} sowie der OCXO~\ref{sct:Oscillator} mit einem Laptop verbunden. Zu sehen ist ebenfalls die SDRangel Software, die zum Aufzeichnen verwendet wird. Die Antennen sind so ausgerichtet, dass eine in Richtung der Einflugsschneise des Flughafens, die andere in Richtung des Fernsehturms zeigt. Die Ausrichtung der Antennen ist in Abbildung~\ref{fig:Einflugschneise} gezeigt. Zu beginn jeder Messung werden die Antennen parallel auf den Fernsehturm gerichtet. Nach dem einige Sekunden aufgenommen sind, wird die Antenne des Überwachungskanals in Richtung Flughafen gedreht. Somit soll garantiert werden, dass auf beiden Kanälen ein starkes 5G Signal zur späteren Synchronisierung vorliegt.

\begin{figure}
    \centering
    \includegraphics[width=\textwidth]{images/Messaufbau.jpg}
    \caption{Aufbau der Messung}\label{fig:Messaufbau}
\end{figure}
