% !TeX root = ../report.tex
\begin{abstract}
    Mit dem Rückgang konventioneller Broadcast-Systeme wie FM und DVB-T in einigen Ländern ist die Suche nach weiträumigen terrestrischen Beleuchtern für Passivradar wiederbelebt. In dieser Studienarbeit wird ein experimenteller Passivradar-Aufbau mit realem 5G Broadcast Beleuchter beschrieben. Damit sollen startende und landende Flugzeuge detektiert werden. Die Arbeit umfasst sowohl Hardware als auch Software eines selbst entwickelten Systems. Kernelement des Aufbaus sind zwei günstige ADALM Pluto Software-Defined-Radios von Analog Devices. Die komplette Signalverarbeitungskette wird besprochen und die in Feldmessungen erworbenen Kenntnisse diskutiert.

    Trotz allen Anstrengungen konnte bis Abschluss der Projektarbeit keine funktionale Detektionskette realisiert werden. Als Mitursache wurde ein Mangel an allgemeinen Referenzmaterialien zu LTE-basiertem 5G Broadcast sowie konkreten Informationen zur betrachteten Sendeanlage identifiziert.
\end{abstract}
